\documentclass[10pt,a4paper]{article}
\usepackage{listings}
\usepackage{hyperref}
\usepackage{graphicx}
\usepackage{float}
\usepackage{subcaption}
\usepackage{cleveref}
\usepackage{amsmath}
\usepackage{enumitem}

\newcommand{\scomment}[1]{{\bf{\color{red}{{[Suyash: #1]}}}}}
\newcommand{\acomment}[1]{{\bf{\color{blue}{{[Aman: #1]}}}}}
\newcommand{\fullref}[1]{\hyperref[{#1}]{\autoref*{#1} - \nameref*{#1}}} % One single link
\newcommand{\figref}[1]{\hyperref[{#1}]{\autoref*{#1}}} % One single link

\hypersetup{
colorlinks=true,
allcolors=blue,
}


\usepackage{geometry}
\geometry{
a4paper,
left=20mm,
top=30mm,
}

\title{COL772: Project Proposal}
\author{Suyash Agrawal (2015CS10262) \\ Aman Agrawal (2015CS10210)}

\begin{document}
\maketitle
\section{Goal}
    Our aim is to create a model that can learn to auto reply to tweets.
\section{Dataset}
\section{Literature}
    Recently, google released a paper ``Smart Reply''\cite{paper:smartReply}, which is their model for 
    email auto responder. Apart from this, there are various papers on summarization\cite{paper:pointer}
    where we plan to take inspiration from.\\
    We have not found a paper that exactly does twitter replies and thus we would be writing code from scratch.
\section{Evaluation}
    For quantitative evaluation, we are currently planning on using the perplexity as our loss measure. The gold
    labels, would be a held out test data from the whole dataset. We would also be reporting the BLEU, METEOR and
    ROGUE metric, of our predictions.\\
    For qualitative evaluation, we can just look at some randomly sampled results and judge their quality.

    \bibliographystyle{unsrt}
    \bibliography{proposal}
\end{document}
